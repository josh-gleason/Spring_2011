\documentclass[12pt,a4paper,oneside]{article}
\special{papersize=8.5in,11in}

\setlength{\textwidth}{6.5in} %1 in margins
\setlength{\textheight}{9.75in}
\setlength{\oddsidemargin}{0in}
\setlength{\topmargin}{0in}
\setlength{\headheight}{0in}
\setlength{\headsep}{0in}
\setlength{\marginparwidth}{1in}
\setlength{\footskip}{0.25in}
\setlength{\marginparsep}{0in}

\renewcommand{\abstractname}{}

%\usepackage{array}
%\usepackage{amssymb,amsmath}
%\usepackage{multirow}
\usepackage{subfigure}
\usepackage{graphicx}
%\usepackage{graphics}
\usepackage{color}
%\usepackage{fancyhdr}
\usepackage{caption}
\usepackage{color}
\usepackage{listings}
%\usepackage{courier}
%\usepackage{cite}
%\usepackage[section]{placeins}
\usepackage[pdftex,bookmarks=true,colorlinks=true,linkcolor=blue]{hyperref}
%\usepackage[all]{hypcap}

\lstset{
  basicstyle=\scriptsize\ttfamily, % Standardschrift
    %numbers=left,               % Ort der Zeilennummern
    numberstyle=\tiny,          % Stil der Zeilennummern
    %stepnumber=2,               % Abstand zwischen den Zeilennummern
    numbersep=1pt,              % Abstand der Nummern zum Text
    tabsize=2,                  % Groesse von Tabs
    extendedchars=true,         %
    breaklines=true,            % Zeilen werden Umgebrochen
    keywordstyle=\color{red},
    frame=b,         
    %        keywordstyle=[1]\textbf,    % Stil der Keywords
      %        keywordstyle=[2]\textbf,    %
      %        keywordstyle=[3]\textbf,    %
      %        keywordstyle=[4]\textbf,   \sqrt{\sqrt{}} %
      stringstyle=\color{white}\ttfamily, % Farbe der String
      showspaces=false,           % Leerzeichen anzeigen ?
      showtabs=false,             % Tabs anzeigen ?
      xleftmargin=17pt,
    framexleftmargin=17pt,
    framexrightmargin=5pt,
    framexbottommargin=4pt,
    %backgroundcolor=\color{lightgray},
    showstringspaces=false      % Leerzeichen in Strings anzeigen ?        
}

\lstloadlanguages{% Check Dokumentation for further languages ...
  %[Visual]Basic
    %Pascal
    %C
    C++
    %XML
    %HTML
    %Java
}

%\DeclareCaptionFont{blue}{\color{blue}} 

%\captionsetup[lstlisting]{singlelinecheck=false, labelfont={blue}, textfont={blue}}
\usepackage{caption}
\DeclareCaptionFont{white}{\color{white}}
\DeclareCaptionFormat{listing}{\colorbox[cmyk]{0.43, 0.35, 0.35,0.01}{\parbox{\textwidth}{\hspace{15pt}#1#2#3}}}
\captionsetup[lstlisting]{format=listing,labelfont=white,textfont=white, singlelinecheck=false, margin=0pt, font={bf,footnotesize}}

\newcommand{\HRule}{\rule{\linewidth}{0.5mm}}

\makeatletter
\renewcommand\l@section[2]{%
  \ifnum \c@tocdepth >\z@
    \addpenalty\@secpenalty
    \addvspace{1.0em \@plus\p@}%
    \setlength\@tempdima{1.5em}%
    \begingroup
      \parindent \z@ \rightskip \@pnumwidth
      \parfillskip -\@pnumwidth
      \leavevmode \bfseries
      \advance\leftskip\@tempdima
      \hskip -\leftskip
      #1\nobreak\
      \leaders\hbox{$\m@th\mkern \@dotsep mu\hbox{.}\mkern \@dotsep mu$}
     \hfil \nobreak\hb@xt@\@pnumwidth{\hss #2}\par
    \endgroup
  \fi}
\makeatother

