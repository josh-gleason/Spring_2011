%\begin{figure}[hbt]
%  \centering
%  \label{fig:}
%  \subfigure[CAP]{
%    \includegraphics[width=0.4\textwidth]{}
%  }
%  \caption{}
%\end{figure}
\newcolumntype{S}{>{\centering\arraybackslash} m{0.26\textwidth} }

~\vfill

\begin{figure}[hbt]
  \centering
  \label{fig:smooth}
  \subfigure[lenna 5x5]{
    \includegraphics[height=0.22\textwidth]{../smooth/lenna5.jpg}
  } 
  \subfigure[sf 5x5]{
    \includegraphics[height=0.22\textwidth]{../smooth/sf5.jpg}
  } 
  \subfigure[lenna 5x5 OpenCV]{
    \includegraphics[height=0.22\textwidth]{../smooth/lenna5ocv.jpg}
  }
  \subfigure[sf 5x5 OpenCV]{
    \includegraphics[height=0.22\textwidth]{../smooth/sf5ocv.jpg}
  }  \\
  \subfigure[lenna 11x11]{
    \includegraphics[height=0.22\textwidth]{../smooth/lenna11.jpg}
  } 
  \subfigure[sf 11x11]{
    \includegraphics[height=0.22\textwidth]{../smooth/sf11.jpg}
  } 
  \subfigure[lenna 11x11 OpenCV]{
    \includegraphics[height=0.22\textwidth]{../smooth/lenna11ocv.jpg}
  } 
  \subfigure[sf 11x11 OpenCV]{
    \includegraphics[height=0.22\textwidth]{../smooth/sf11ocv.jpg}
  }  \\
  \caption{Smoothing using separable Gaussian blur comparing to OpenCV's GaussianBlur function.}
\end{figure}

\vfill

~\vfill

\begin{figure}[hbt]
	\centering
	\label{fig:lennagauss}
  \subfigure[lenna A Lvl 1]{
    \includegraphics[height=0.22\textwidth]{../gauss/lennaA_0.jpg}
  } 
  \subfigure[lenna B Lvl 1]{
    \includegraphics[height=0.22\textwidth]{../gauss/lennaB_0.jpg}
  } 
  \subfigure[lenna A Lvl 2]{
    \includegraphics[height=0.22\textwidth]{../gauss/lennaA_1.jpg}
  } 
  \subfigure[lenna B Lvl 2]{
    \includegraphics[height=0.22\textwidth]{../gauss/lennaB_1.jpg}
  } \\
  \subfigure[lenna A Lvl 3]{
    \includegraphics[height=0.22\textwidth]{../gauss/lennaA_2.jpg}
  } 
  \subfigure[lenna B Lvl 3]{
    \includegraphics[height=0.22\textwidth]{../gauss/lennaB_2.jpg}
  } 
  \subfigure[lenna A Lvl 4]{
    \includegraphics[height=0.22\textwidth]{../gauss/lennaA_3.jpg}
  } 
  \subfigure[lenna B Lvl 4]{
    \includegraphics[height=0.22\textwidth]{../gauss/lennaB_3.jpg}
  } \\
  \subfigure[lenna A Lvl 5]{
    \includegraphics[height=0.22\textwidth]{../gauss/lennaA_4.jpg}
  } 
  \subfigure[lenna B Lvl 5]{
    \includegraphics[height=0.22\textwidth]{../gauss/lennaB_4.jpg}
  } 
  \subfigure[lenna A Lvl 6]{
    \includegraphics[height=0.22\textwidth]{../gauss/lennaA_5.jpg}
  } 
  \subfigure[lenna B Lvl 6]{
    \includegraphics[height=0.22\textwidth]{../gauss/lennaB_5.jpg}
  } \\
	\caption{Comparison of Gaussian Pyramids Using Method A and B on lenna.pgm.}
\end{figure}

\vfill

~\vfill

\begin{figure}[hbt]
	\centering
	\label{fig:sfgauss}
  \subfigure[sf A Lvl 1]{
    \includegraphics[height=0.22\textwidth]{../gauss/sfA_0.jpg}
  } 
  \subfigure[sf B Lvl 1]{
    \includegraphics[height=0.22\textwidth]{../gauss/sfB_0.jpg}
  } 
  \subfigure[sf A Lvl 2]{
    \includegraphics[height=0.22\textwidth]{../gauss/sfA_1.jpg}
  } 
  \subfigure[sf B Lvl 2]{
    \includegraphics[height=0.22\textwidth]{../gauss/sfB_1.jpg}
  } \\
  \subfigure[sf A Lvl 3]{
    \includegraphics[height=0.22\textwidth]{../gauss/sfA_2.jpg}
  } 
  \subfigure[sf B Lvl 3]{
    \includegraphics[height=0.22\textwidth]{../gauss/sfB_2.jpg}
  } 
  \subfigure[sf A Lvl 4]{
    \includegraphics[height=0.22\textwidth]{../gauss/sfA_3.jpg}
  } 
  \subfigure[sf B Lvl 4]{
    \includegraphics[height=0.22\textwidth]{../gauss/sfB_3.jpg}
  } \\
  \subfigure[sf A Lvl 5]{
    \includegraphics[height=0.22\textwidth]{../gauss/sfA_4.jpg}
  } 
  \subfigure[sf B Lvl 5]{
    \includegraphics[height=0.22\textwidth]{../gauss/sfB_4.jpg}
  } 
  \subfigure[sf A Lvl 6]{
    \includegraphics[height=0.22\textwidth]{../gauss/sfA_5.jpg}
  } 
  \subfigure[sf B Lvl 6]{
    \includegraphics[height=0.22\textwidth]{../gauss/sfB_5.jpg}
  } \\
	\caption{Comparison of Gaussian Pyramids Using Method A and B on sf.pgm.}
\end{figure}

\vfill

~\vfill

\begin{figure}[hbt]
	\centering
	\label{fig:lennalaplacian}
  \subfigure[lenna A Lvl 1]{
    \includegraphics[height=0.22\textwidth]{../laplacian/lennaA_0.jpg}
  } 
  \subfigure[lenna B Lvl 1]{
    \includegraphics[height=0.22\textwidth]{../laplacian/lennaB_0.jpg}
  } 
  \subfigure[lenna A Lvl 2]{
    \includegraphics[height=0.22\textwidth]{../laplacian/lennaA_1.jpg}
  } 
  \subfigure[lenna B Lvl 2]{
    \includegraphics[height=0.22\textwidth]{../laplacian/lennaB_1.jpg}
  } \\
  \subfigure[lenna A Lvl 3]{
    \includegraphics[height=0.22\textwidth]{../laplacian/lennaA_2.jpg}
  } 
  \subfigure[lenna B Lvl 3]{
    \includegraphics[height=0.22\textwidth]{../laplacian/lennaB_2.jpg}
  } 
  \subfigure[lenna A Lvl 4]{
    \includegraphics[height=0.22\textwidth]{../laplacian/lennaA_3.jpg}
  } 
  \subfigure[lenna B Lvl 4]{
    \includegraphics[height=0.22\textwidth]{../laplacian/lennaB_3.jpg}
  } \\
  \subfigure[lenna A Lvl 5]{
    \includegraphics[height=0.22\textwidth]{../laplacian/lennaA_4.jpg}
  } 
  \subfigure[lenna B Lvl 5]{
    \includegraphics[height=0.22\textwidth]{../laplacian/lennaB_4.jpg}
  } \\
	\caption{Comparison of Laplacian Pyramids Using Method A and B on lenna.pgm.}
\end{figure}

\vfill

~\vfill

\begin{figure}[hbt]
	\centering
	\label{fig:sflaplacian}
  \subfigure[sf A Lvl 1]{
    \includegraphics[height=0.22\textwidth]{../laplacian/sfA_0.jpg}
  } 
  \subfigure[sf B Lvl 1]{
    \includegraphics[height=0.22\textwidth]{../laplacian/sfB_0.jpg}
  } 
  \subfigure[sf A Lvl 2]{
    \includegraphics[height=0.22\textwidth]{../laplacian/sfA_1.jpg}
  } 
  \subfigure[sf B Lvl 2]{
    \includegraphics[height=0.22\textwidth]{../laplacian/sfB_1.jpg}
  } \\
  \subfigure[sf A Lvl 3]{
    \includegraphics[height=0.22\textwidth]{../laplacian/sfA_2.jpg}
  } 
  \subfigure[sf B Lvl 3]{
    \includegraphics[height=0.22\textwidth]{../laplacian/sfB_2.jpg}
  } 
  \subfigure[sf A Lvl 4]{
    \includegraphics[height=0.22\textwidth]{../laplacian/sfA_3.jpg}
  } 
  \subfigure[sf B Lvl 4]{
    \includegraphics[height=0.22\textwidth]{../laplacian/sfB_3.jpg}
  } \\
  \subfigure[sf A Lvl 5]{
    \includegraphics[height=0.22\textwidth]{../laplacian/sfA_4.jpg}
  } 
  \subfigure[sf B Lvl 5]{
    \includegraphics[height=0.22\textwidth]{../laplacian/sfB_4.jpg}
  } \\
	\caption{Comparison of Laplacian Pyramids Using Method A and B on sf.pgm.}
\end{figure}

\vfill

~\vfill

\begin{figure}[hbt]
  \centering
  \label{fig:edgeslenna1}
  \subfigure[lenna Lvl 1]{
    \includegraphics[height=0.22\textwidth]{../edges/lenna_1_0.jpg}
  }
  \subfigure[lenna Lvl 2]{
    \includegraphics[height=0.22\textwidth]{../edges/lenna_1_1.jpg}
  }
  \subfigure[lenna Lvl 3]{
    \includegraphics[height=0.22\textwidth]{../edges/lenna_1_2.jpg}
  }
  \subfigure[lenna Lvl 4]{
    \includegraphics[height=0.22\textwidth]{../edges/lenna_1_3.jpg}
  }
  \subfigure[lenna Lvl 5]{
    \includegraphics[height=0.22\textwidth]{../edges/lenna_1_4.jpg}
  }
  \caption{Edges on lenna detected by second derivative, threshold value = 1}
\end{figure}

\begin{figure}[hbt]
  \centering
  \label{fig:edgeslenna4}
  \subfigure[lenna Lvl 1]{
    \includegraphics[height=0.22\textwidth]{../edges/lenna_4_0.jpg}
  }
  \subfigure[lenna Lvl 2]{
    \includegraphics[height=0.22\textwidth]{../edges/lenna_4_1.jpg}
  }
  \subfigure[lenna Lvl 3]{
    \includegraphics[height=0.22\textwidth]{../edges/lenna_4_2.jpg}
  }
  \subfigure[lenna Lvl 4]{
    \includegraphics[height=0.22\textwidth]{../edges/lenna_4_3.jpg}
  }
  \subfigure[lenna Lvl 5]{
    \includegraphics[height=0.22\textwidth]{../edges/lenna_4_4.jpg}
  }
  \caption{Edges on lenna detected by second derivative, threshold value = 4}
\end{figure}

\vfill

~\vfill

\begin{figure}[hbt]
  \centering
  \label{fig:edgessf1}
  \subfigure[sf Lvl 1]{
    \includegraphics[height=0.22\textwidth]{../edges/sf_1_0.jpg}
  }
  \subfigure[sf Lvl 2]{
    \includegraphics[height=0.22\textwidth]{../edges/sf_1_1.jpg}
  }
  \subfigure[sf Lvl 3]{
    \includegraphics[height=0.22\textwidth]{../edges/sf_1_2.jpg}
  }
  \subfigure[sf Lvl 4]{
    \includegraphics[height=0.22\textwidth]{../edges/sf_1_3.jpg}
  }
  \subfigure[sf Lvl 5]{
    \includegraphics[height=0.22\textwidth]{../edges/sf_1_4.jpg}
  }
  \caption{Edges on sf detected by second derivative, threshold value = 1}
\end{figure}

\begin{figure}[hbt]
  \centering
  \label{fig:edgessf4}
  \subfigure[sf Lvl 1]{
    \includegraphics[height=0.22\textwidth]{../edges/sf_4_0.jpg}
  }
  \subfigure[sf Lvl 2]{
    \includegraphics[height=0.22\textwidth]{../edges/sf_4_1.jpg}
  }
  \subfigure[sf Lvl 3]{
    \includegraphics[height=0.22\textwidth]{../edges/sf_4_2.jpg}
  }
  \subfigure[sf Lvl 4]{
    \includegraphics[height=0.22\textwidth]{../edges/sf_4_3.jpg}
  }
  \subfigure[sf Lvl 5]{
    \includegraphics[height=0.22\textwidth]{../edges/sf_4_4.jpg}
  }
  \caption{Edges on sf detected by second derivative, threshold value = 4}
\end{figure}

~\vfill

\vfill

\begin{figure}[hbt]
  \centering
  \label{fig:aerialsobel}
  \subfigure[aerial Sobel X]{
    \includegraphics[height=0.22\textwidth]{../sobel/aerial_50_sobelx.jpg}
  }
  \subfigure[aerial Sobel Y]{
    \includegraphics[height=0.22\textwidth]{../sobel/aerial_50_sobely.jpg}
  }
  \subfigure[aerial Magnitude]{
    \includegraphics[height=0.22\textwidth]{../sobel/aerial_50_mag.jpg}
  }
  \subfigure[aerial Direction]{
    \includegraphics[height=0.22\textwidth]{../sobel/aerial_50_dir.jpg}
  } \\
  \subfigure[Threshold = 50]{
    \includegraphics[height=0.22\textwidth]{../sobel/aerial_50_edges.jpg}
  }
  \subfigure[Threshold = 100]{
    \includegraphics[height=0.22\textwidth]{../sobel/aerial_100_edges.jpg}
  }
  \subfigure[Threshold = 200]{
    \includegraphics[height=0.22\textwidth]{../sobel/aerial_200_edges.jpg}
  }
  \subfigure[Threshold = 300]{
    \includegraphics[height=0.22\textwidth]{../sobel/aerial_300_edges.jpg}
  }
  \caption{Sobel operations on aerial.pgm as well as the magnitude thresholded at various levels.}
\end{figure}

~\vfill

\vfill

\begin{figure}[hbt]
  \centering
  \label{fig:wheelsobel}
  \subfigure[wheel Sobel X]{
    \includegraphics[height=0.22\textwidth]{../sobel/wheel_50_sobelx.jpg}
  }
  \subfigure[wheel Sobel Y]{
    \includegraphics[height=0.22\textwidth]{../sobel/wheel_50_sobely.jpg}
  }
  \subfigure[wheel Magnitude]{
    \includegraphics[height=0.22\textwidth]{../sobel/wheel_50_mag.jpg}
  }
  \subfigure[wheel Direction]{
    \includegraphics[height=0.22\textwidth]{../sobel/wheel_50_dir.jpg}
  } \\
  \subfigure[Threshold = 50]{
    \includegraphics[height=0.22\textwidth]{../sobel/wheel_50_edges.jpg}
  }
  \subfigure[Threshold = 100]{
    \includegraphics[height=0.22\textwidth]{../sobel/wheel_100_edges.jpg}
  }
  \subfigure[Threshold = 200]{
    \includegraphics[height=0.22\textwidth]{../sobel/wheel_200_edges.jpg}
  }
  \subfigure[Threshold = 400]{
    \includegraphics[height=0.22\textwidth]{../sobel/wheel_400_edges.jpg}
  }
  \caption{Sobel operations on wheel.pgm as well as the magnitude thresholded at various levels.}
\end{figure}

\vfill

%\vfill
